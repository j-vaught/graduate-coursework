% ============================================================================
% HOMEWORK ASSIGNMENT DOCUMENT
% ============================================================================

\documentclass[12pt]{article}

% ============================================================================
% PACKAGE IMPORTS
% ============================================================================
\usepackage{amsmath}        % Advanced math symbols and equations
\usepackage{amssymb}        % Additional math symbols (like ℝ, ℕ, etc.)
\usepackage{graphicx}       % Insert images/figures into your document
\usepackage{tikz}           % Create diagrams directly in LaTeX
\usepackage[margin=1in]{geometry}
\usepackage{setspace}       % Control line spacing
\usepackage{xcolor}         % Use colors in text and boxes
\usepackage{enumitem}       % Customize list numbering
\usepackage{tcolorbox}      % Colored containers for highlights
\usepackage{fancyhdr}       % Custom headers/footers
\usepackage{lastpage}       % Reference the last page number
\usepackage{listings}       % Code listings
\usepackage{pgfplots}       % Plotting library
% ============================================================================
% USC BRAND COLORS
% ============================================================================
\definecolor{USC_Garnet}{HTML}{73000A}
\definecolor{USC_Sandstorm}{HTML}{FFF2E3}
\definecolor{USC_Black90}{HTML}{565656}
\definecolor{USC_Black10}{HTML}{ECECEC}
\definecolor{USC_Honeycomb}{HTML}{65780B}
\lstset{
  language=Matlab,
  basicstyle=\ttfamily\small,
  keywordstyle=\color{blue},
  commentstyle=\color{USC_Honeycomb},
  stringstyle=\color{USC_Garnet},
  breaklines=true,
  showstringspaces=false,
  backgroundcolor=\color{USC_Black10},
  columns=fullflexible,
  keepspaces=true,
  literate={–}{-}1 {—}{-}1 {…}{...}1 {’}{'}1 {“}{''}1 {”}{''}1
}

% ============================================================================
% CUSTOM TCOLORBOX STYLES
% ============================================================================
\newtcolorbox{stepbox}{
  colback=white,
  colframe=USC_Garnet,
  fonttitle=\bfseries,
  title=Step,
  sharp corners,
  colbacktitle=USC_Garnet,
  coltitle=white
}

\newtcolorbox{codebox}{
  colback=USC_Black10,
  colframe=USC_Black90,
  fonttitle=\bfseries,
  title=MATLAB Implementation,
  sharp corners,
  colbacktitle=USC_Black90,
  coltitle=white
}

\newtcolorbox{resultsbox}{
  colback=white,
  colframe=USC_Honeycomb,
  fonttitle=\bfseries,
  title=Results,
  sharp corners,
  colbacktitle=USC_Honeycomb,
  coltitle=white
}

% ============================================================================
% HEADER AND FOOTER SETUP
% ============================================================================
\pagestyle{fancy}
\setlength{\headheight}{14.5pt}
\fancyhf{}
\fancyhead[L]{Page \thepage\ of \pageref{LastPage}}
\fancyhead[C]{EMCH 721}
\fancyhead[R]{J.C. Vaught}
\renewcommand{\headrulewidth}{0pt}

% ============================================================================
% TIKZ LIBRARIES
% ============================================================================
\usetikzlibrary{shapes.geometric, arrows.meta, positioning, decorations.pathmorphing, patterns, calc}

% ============================================================================
% CUSTOM QUESTION COMMAND
% ============================================================================
\newcommand{\question}[1]{%
  \clearpage
  \vspace{0.5cm}
  {\noindent\normalsize \textbf{#1}}%
  \vspace{0.2cm}
  \hrule
  \vspace{0.1cm}
  \hrule
  \vspace{0.3cm}
}

\newcommand{\solutionframework}{%
\subsection*{Solution Outline}
\begin{stepbox}
\textbf{Planned Steps}\\[0.5em]
Outline your step-by-step approach here.
\end{stepbox}
\subsection*{MATLAB Implementation}
\begin{codebox}
Add MATLAB code or pseudocode here.
\end{codebox}
\subsection*{Results Summary}
\begin{resultsbox}
Summarize key intermediate and final results here.
\end{resultsbox}
}

% ============================================================================
% DOCUMENT INFORMATION
% ============================================================================
\title{EMCH 721: Homework 03}
\author{Instructor: Dr. Victor Giurgiutiu\\Student: J.C. Vaught}
\date{}

% ============================================================================
% BEGIN DOCUMENT
% ============================================================================
\begin{document}

\maketitle
\setlength{\parindent}{0pt}
\setstretch{1.1}

% List formatting defaults
\setlist[enumerate,1]{label=\arabic*.}
\setlist[enumerate,2]{label=\alph*.}

% ============================================================================
% TABLE OF CONTENTS
% ============================================================================
\begin{center}
\begin{tcolorbox}[colback=white, colframe=gray!50!black, colbacktitle=gray!20!white,
                  coltitle=black, sharp corners, boxrule=1pt,
                  title=\Large\bfseries Table of Contents]
\begin{tabular}{p{0.91\textwidth}r}
\textbf{A.1 Flexural Vibration (Cantilever)} \dotfill & \textbf{\pageref{quest:A1}} \\
\textbf{A.2 Torsional Vibration (Cantilever)} \dotfill & \textbf{\pageref{quest:A2}} \\
\textbf{B. Aeroelastic Equations of a Straight Wing} \dotfill & \textbf{\pageref{quest:B}} \\
\textbf{C. Ground Vibration Test (GVT) Simulation} \dotfill & \textbf{\pageref{quest:C}} \\
\textbf{D. Flutter Eigenvalue Analysis} \dotfill & \textbf{\pageref{quest:D}} \\
\textbf{E.1 Flexural Challenge (Extra Credit)} \dotfill & \textbf{\pageref{quest:E1}} \\
\textbf{E.2 Torsional Challenge (Extra Credit)} \dotfill & \textbf{\pageref{quest:E2}} \\
\textbf{Appendix: MATLAB Source Listings} \dotfill & \textbf{\pageref{appendix:matlab}} \\
\end{tabular}
\vspace{0.2cm}
\end{tcolorbox}
\end{center}

\clearpage

% ============================================================================
% PROBLEM CONTEXT
% ============================================================================
\section*{Problem Context}

Consider an aircraft wing undergoing ground and flight testing at airspeed $U$ (Figure~\ref{fig:wing-overview}). The wing is fixed at one root and free at the tip. It has straight elastic, mass, and aerodynamic axes located at points $P$, $C$, and $Q$, respectively. The mass offset from $P$ to $C$ is $x_{CP}$; the aerodynamic offset from $P$ to $Q$ is $x_{QP}$. The wing executes small-amplitude plunge $w(z,t)$ (positive downward) and pitch $\phi(z,t)$ (positive nose-up, clockwise) oscillations.

\begin{figure}[htbp]
  \centering
  \providecolor{UofSCGarnet}{RGB}{115, 0, 10}
\providecolor{UofSCBlack}{RGB}{0, 0, 0}
\providecolor{UofSCWhite}{RGB}{255, 255, 255}
\providecolor{UofSC90Black}{RGB}{54, 54, 54}
\providecolor{UofSC70Black}{RGB}{92, 92, 92}
\providecolor{UofSC50Black}{RGB}{162, 162, 162}
\providecolor{UofSC30Black}{RGB}{199, 199, 199}
\providecolor{UofSC10Black}{RGB}{235, 235, 235}
\providecolor{UofSCWarmGrey}{RGB}{103, 97, 86}
\providecolor{UofSCSandstorm}{RGB}{255, 242, 227}
\providecolor{UofSCRose}{RGB}{204, 46, 64}
\providecolor{UofSCAtlantic}{RGB}{70, 106, 159}
\providecolor{UofSCCongaree}{RGB}{31, 65, 77}
\providecolor{UofSCHorseshoe}{RGB}{101, 120, 11}
\providecolor{UofSCGrass}{RGB}{206, 211, 24}
\providecolor{UofSCHoneycomb}{RGB}{164, 145, 55}
\providecolor{UofSCDarkGarnet}{RGB}{87, 0, 8}
\providecolor{UofSCAzalea}{RGB}{132, 66, 71}

\newcommand{\IsoViewDiagram}{%
    \begin{tikzpicture}[x={(0.866cm,-0.4cm)}, y={(0.6cm,0.5cm)}, z={(0cm,1cm)}, scale=1.3]

        \def\span{6}
        \def\chord{3}

        \fill[UofSCWarmGrey!60] (0, -1, -2) -- (0, \chord+1, -2) -- (0, \chord+1, 2) -- (0, -1, 2) -- cycle;
        \draw[thick] (0, -1, -2) -- (0, \chord+1, -2) -- (0, \chord+1, 2) -- (0, -1, 2) -- cycle;

        \newcommand{\drawAirfoilIso}[1]{%
            plot [smooth cycle, tension=0.6] coordinates {
                (##1, 0, 0)
                (##1, 0.2*\chord, 0.1*\chord)
                (##1, 0.6*\chord, 0.08*\chord)
                (##1, \chord, 0)
                (##1, 0.6*\chord, -0.05*\chord)
                (##1, 0.2*\chord, -0.08*\chord)
            }
        }

        \draw[thick, UofSCBlack] (0, \chord, 0) -- (\span, \chord, 0);
        \draw[thick, UofSCBlack] (0, 0, 0) -- (\span, 0, 0);
        \draw[thick, fill=UofSC10Black] \drawAirfoilIso{\span};
        \draw[dashed, thick] \drawAirfoilIso{0};

        \coordinate (RootQ) at (0, 0.2*\chord, 0);
        \coordinate (TipQ) at (\span, 0.2*\chord, 0);
        \draw[densely dotted, UofSCAtlantic, thick] (RootQ) -- (TipQ);
        \node[right, font=\footnotesize, UofSCAtlantic] at (\span+0.3, 0.2*\chord, 0) {$Q_{\text{axis}}$};
        \fill[UofSCAtlantic] (TipQ) circle (1pt);

        \coordinate (RootC) at (0, 0.4*\chord, 0);
        \coordinate (TipC) at (\span, 0.4*\chord, 0);
        \draw[densely dotted, UofSCBlack, thick] (RootC) -- (TipC);
        \node[right, font=\footnotesize, UofSCBlack] at (\span+0.3, 0.4*\chord, 0) {$C_{\text{axis}}$};
        \fill[UofSCBlack] (TipC) circle (1pt);

        \coordinate (RootP) at (0, 0.6*\chord, 0);
        \coordinate (TipP) at (\span, 0.6*\chord, 0);
        \draw[dashed, UofSCGarnet, thick] (RootP) -- (TipP);
        \node[right, font=\footnotesize, UofSCGarnet] at (\span+0.3, 0.6*\chord, 0) {$P_{\text{axis}}$};
        \fill[UofSCGarnet] (TipP) circle (1pt);

        \fill[UofSCAtlantic!40, opacity=0.9]
            plot [smooth, tension=0.6] coordinates {(0,0,0) (0, 0.2*\chord, 0.1*\chord) (0, 0.6*\chord, 0.08*\chord) (0,\chord,0)}
            --
            plot [smooth, tension=0.6] coordinates {(\span,\chord,0) (\span, 0.6*\chord, 0.08*\chord) (\span, 0.2*\chord, 0.1*\chord) (\span,0,0)}
            -- cycle;

        \draw[thick] (0,0,0) -- (\span,0,0);
        \draw[thick] (0,\chord,0) -- (\span,\chord,0);
        \draw[thick] \drawAirfoilIso{\span};

        \draw[->, thick, UofSCGarnet] (0, \chord, 0) -- (\span + 0.5, \chord, 0) node[right] {$z$ (Span)};

        \foreach \x in {1, 2, 3, 4, 5} {
            \draw[->, ultra thick, UofSCGarnet] (\x, -1.2, 0) -- (\x, -0.2, 0);
        }
        \node[right, UofSCGarnet] at (3, -2.0, 0) {$U$};

        \node[align=center] at (\span+0.5, \chord+0.5, 0.5) {Tip (Free)};
        \node[align=center] at (-0.5, \chord/2, 2.5) {Root (Fixed)};
    \end{tikzpicture}%
}

\providecolor{UofSCGarnet}{RGB}{115, 0, 10}
\providecolor{UofSCBlack}{RGB}{0, 0, 0}
\providecolor{UofSCAtlantic}{RGB}{70, 106, 159}
\providecolor{UofSCLightGray}{RGB}{200, 200, 200}

\newcommand{\CrossSectionDiagram}{%
    \begin{tikzpicture}[scale=1.5, >=Latex]
        \draw[thick, fill=UofSCLightGray] plot [smooth cycle, tension=0.6] coordinates {
            (0,0)
            (1.2, 0.6)
            (3.6, 0.48)
            (6,0)
            (3.6, -0.3)
            (1.2, -0.48)
        };

        \draw[dashed, UofSCBlack] (-0.2,0) -- (6.5,0) node[right] {Reference Chord Line};

        \coordinate (Q) at (1.2,0);
        \coordinate (C) at (2.4,0);
        \coordinate (P) at (3.6,0);

        \draw[<->, thick] ($(Q)+(0,1)$) -- ($(P)+(0,1)$) node[midway, fill=white] {$x_{QP}$};
        \draw[dashed] (P) -- ($(P)+(0,1.2)$);
        \draw[<->, thick] ($(P)+(0,0.75)$) -- ($(C)+(0,0.75)$) node[midway, fill=white] {$x_{CP}$};
        \draw[dashed] (C) -- ($(C)+(0,0.8)$);

        \fill[UofSCGarnet] (P) circle (2pt) node[below=7mm] {P};
        \fill[UofSCBlack] (C) circle (2pt) node[below=7mm] {C};
        \fill[UofSCAtlantic] (Q) circle (2pt) node[below=7mm] {Q};

        \draw[->, ultra thick, UofSCAtlantic] (-2, 0) -- (-0.5, 0) node[above] {$U$};

        \draw[->, ultra thick, UofSCAtlantic] (Q) -- ($(Q)+(0,1.5)$) node[right] {$L(U)$};

        \draw[->, ultra thick, UofSCGarnet] ($(Q)+(-2,-0.4)$) arc (270:90:0.4) node[midway, above left] {$\theta(z,t)$};
    \end{tikzpicture}%
}


\IsoViewDiagram

\vspace{1em}

\CrossSectionDiagram

  \caption{Overview of the rectangular test wing with spanwise axes $P$, $C$, and $Q$, and the incoming flow $U$. Cross-section shows offsets $x_{CP}$ and $x_{QP}$.}
  \label{fig:wing-overview}
\end{figure}


\subsection*{Notes}
\begin{tcolorbox}[colback=white,colframe=black,colbacktitle=gray!20!white,
                  coltitle=black,sharp corners,boxrule=1pt,
                  title=Submission Notes]
\begin{itemize}[leftmargin=1.2em]
  \item Always display input data.
  \item Show work and relevant comments.
  \item Attach MATLAB codes to receive partial credit where applicable.
  \item Report numerical results to three significant digits (use four if the first digit is 1).
  \item Problems solved beyond the minimum requirements may receive bonus points.
  \item Challenge problems are optional and count toward extra credit.
\end{itemize}
\end{tcolorbox}

\subsection*{Input Data}
\begin{tcolorbox}[colback=white,colframe=black,colbacktitle=gray!20!white,
                  coltitle=black,sharp corners,boxrule=1pt,
                  title=Input Data Summary]
\begin{tabular}{@{}ll@{}}
\textbf{Parameter} & \textbf{Value} \\
Air density $\rho$ & $1.225~\mathrm{kg/m^3}$ \\
Chord $c$ & $0.45~\mathrm{m}$ \\
Wingspan $l$ & $2.5~\mathrm{m}$ \\
Mass $m$ & $3.2~\mathrm{kg}$ \\
Mass moment of inertia $I_0$ & $0.055~\mathrm{kg\cdot m^2}$ \\
Plunge frequency $f_h$ & $1.8~\mathrm{Hz}$ \\
Pitch frequency $f_\theta$ & $5.3~\mathrm{Hz}$ \\
Static offset $x_{CP}$ & $-10\%\,c$ \\
Aerodynamic offset $x_{QP}$ & $35\%\,c$ \\
\end{tabular}
\end{tcolorbox}

% ============================================================================
% QUESTIONS
% ============================================================================

\question{A.1 Flexural vibration of a cantilever beam of length $l$, with section properties $EI$, $m$, $I_0$}\label{quest:A1}
\begin{enumerate}[label=(\alph*)]
  \item Recall the formulae for flexural frequencies and modeshapes of a fixed-free beam.
  \item Calculate and display the roots $\gamma l$ of the characteristic equation for $N_w=4$.
  \item Calculate and display the flexural stiffness $EI$ such that the fundamental flexural frequency matches the plunge frequency of the rigid wing section and display the resulting $EI$ value and units.
  \item Calculate and display the wavenumbers, natural frequencies in rad/s and Hz.
  \item Calculate and plot the modeshapes.
\end{enumerate}

\begin{stepbox}
\textbf{Step 1: Governing ODE (Euler--Bernoulli)}\\
Since a uniform, slender beam obeys the small-deflection Euler--Bernoulli equation, separating variables with $w(x,t)=\phi(x)e^{i\omega t}$ turns time derivatives into the factor $-\omega^2$, giving
\[
EI\,\phi''''(x)=m\,\omega^2\,\phi(x)
\]
and therefore the spatial ODE
\[
\phi''''-\delta^4\phi=0,\qquad \delta^4=\frac{m\omega^2}{EI}.
\]
\end{stepbox}

\begin{stepbox}
\textbf{Step 2: Boundary conditions (fixed–free)}\\
Since the root is clamped, displacement and slope vanish at $x=0$:
\[
\phi(0)=0,\qquad \phi'(0)=0.
\]
Because the tip is free, bending moment and shear must vanish at $x=l$:
\[
\phi''(l)=0,\qquad \phi'''(l)=0.
\]
Applying these four conditions to the general solution enforces the characteristic relation
\[
\cos\beta_j\cosh\beta_j=-1,\quad \beta_j=\delta_j l.
\]
\end{stepbox}

\begin{stepbox}
\textbf{Step 3: Mode-shape coefficients (from tip compatibility)}\\
Since the two free-end conditions couple the integration constants, solving them yields
\[
B_j=\frac{\sinh\beta_j-\sin\beta_j}{\cosh\beta_j+\cos\beta_j},\qquad
A_j=\frac{1}{\sqrt{l}} \ \text{(chosen for unit modal mass).}
\]
\end{stepbox}

\begin{resultsbox}
\textbf{Flexural natural frequencies}\\[0.4em]
Roots of \(\cos\beta\cosh\beta=-1\) give \(\beta_1=1.875,\ \beta_2=4.694,\ \beta_3=7.855,\ \beta_4=10.996,\dots\)\\[0.6em]
\[
\omega_j = \beta_j^2\,\sqrt{\frac{EI}{m\,l^4}}, \qquad
f_j=\frac{\omega_j}{2\pi}.
\]
\end{resultsbox}

\begin{resultsbox}
\textbf{Flexural modeshapes}\\[0.4em]
\[
\phi_j(x)=A_j\Big[(\cosh\delta_j x-\cos\delta_j x)-B_j(\sinh\delta_j x-\sin\delta_j x)\Big],
\]
\[
\delta_j=\frac{\beta_j}{l},\qquad
B_j=\frac{\sinh\beta_j-\sin\beta_j}{\cosh\beta_j+\cos\beta_j},\qquad
A_j=\frac{1}{\sqrt{l}}.
\]
The first four \(\beta_j\) above provide the modeshapes for the required \(N_w=4\).
\end{resultsbox}

\begin{resultsbox}
\centering
\includegraphics[width=\linewidth]{figures/A1_summary.png}

\vspace{0.5em}
\small\textit{Shows A1(b)--(d): characteristic roots for $N_w=4$, the matched flexural stiffness $EI$, flexural wavenumbers, and the corresponding natural frequencies in rad/s and Hz.}
\end{resultsbox}


\begin{resultsbox}
\centering
\includegraphics[width=\linewidth]{figures/A1.e.png}

\vspace{0.5em}
\small\textit{Part A.1(e): plot of the first four cantilever flexural modeshapes used in the A.1(e) discussion.}
\end{resultsbox}

\question{A.2 Torsional vibration of a cantilever beam of length $l$, with section properties $GJ$, $m$, $I_0$}\label{quest:A2}
\begin{enumerate}[label=(\alph*)]
  \item Recall the formulae for torsional frequencies and modeshapes of a fixed-free beam.
  \item Calculate and display the roots $\gamma l$ of the characteristic equation for $N_\phi=3$.
  \item Calculate and display the torsional stiffness $GJ$ such that the fundamental torsional frequency matches the pitch frequency of the rigid wing section and display the resulting $GJ$ value and units.
  \item Calculate and display the wavenumbers, natural frequencies in rad/s and Hz.
  \item Calculate and plot the modeshapes.
\end{enumerate}

\begin{stepbox}
\textbf{Step 1: Governing torsion ODE}\\
Since uniform Saint--Venant torsion transmits torque as \(GJ\,\phi'(z,t)\), equilibrium of a spanwise slice gives
\[
GJ\,\phi''(z,t)=I_0\,\ddot{\phi}(z,t).
\]
Assuming \(\phi(z,t)=\Phi(z)e^{i\omega t}\) yields the spatial equation
\[
\Phi''(z)+\gamma^2 \Phi(z)=0,
\qquad
\gamma^2=\frac{I_0\omega^2}{GJ}.
\]
\end{stepbox}

\begin{stepbox}
\textbf{Step 2: Fixed–free boundary conditions}\\
Since the root is clamped, the twist vanishes:
\[
\Phi(0)=0.
\]
Since the tip is free (zero applied torque), the Saint--Venant torque must vanish:
\[
GJ\,\Phi'(l)=0 \ \Rightarrow\ \Phi'(l)=0.
\]
With the general solution \(\Phi(z)=A\cos\gamma z+B\sin\gamma z\), the clamp sets \(A=0\), and the free-tip condition requires
\[
B\gamma\cos(\gamma l)=0 \ \Rightarrow\ \cos(\gamma l)=0.
\]
Thus the admissible wavenumbers satisfy \(\gamma_j l=(2j-1)\tfrac{\pi}{2}\).
\end{stepbox}

\begin{resultsbox}
\textbf{Torsional natural frequencies}\\[0.4em]
\[
\gamma_j l=\frac{(2j-1)\pi}{2}, \qquad
\omega_j=\gamma_j\sqrt{\frac{GJ}{I_0}}, \qquad
f_j=\frac{\omega_j}{2\pi}=(2j-1)\left(\frac{1}{4l}\right)\sqrt{\frac{GJ}{I_0}}.
\]
\[
\gamma_1 l=1.571,\quad \gamma_2 l=4.712,\quad \gamma_3 l=7.854,\qquad
C_\phi\equiv \sqrt{\tfrac{GJ}{I_0}},\quad
f_j=(2j-1)\frac{C_\phi}{4l}.
\]
\end{resultsbox}

\begin{resultsbox}
\textbf{Torsional modeshapes}\\[0.4em]
\[
\Phi_j(z)=B_j\sin(\gamma_j z), \qquad
B_j=\sqrt{\frac{2}{l}}\ \text{(unit modal mass)}, \qquad
\gamma_j=\frac{(2j-1)\pi}{2l}.
\]
\end{resultsbox}

\begin{resultsbox}
\centering
\includegraphics[width=\linewidth]{figures/A2_summary.png}

\vspace{0.5em}
\small\textit{Shows A.2(b)--(d): characteristic roots for $N_\phi=3$, matched torsional stiffness $GJ$, torsional wavenumbers, and natural frequencies in rad/s and Hz.}
\end{resultsbox}

\begin{resultsbox}
\centering
\includegraphics[width=\linewidth]{figures/A2e.png}

\vspace{0.5em}
\small\textit{Part A.2(e): first three cantilever torsional modeshapes used in the A.2(e) discussion.}
\end{resultsbox}

\question{B. Aeroelastic equations of a straight wing of length $l$, and section properties $EI$, $GJ$, $m$, $I_0$, $x_{CP}$, $x_{QP}$}\label{quest:B}
\begin{enumerate}[label=(\alph*)]
  \item Draw the free body diagram of an infinitesimal element of span $dz$.
  \item Write the plunge and pitch equations of motion as a set of two PDEs in space $z$ and time $t$.
  \item Eliminate the time dependency by assuming solution proportional to $e^{st}$ and cast the equations of motion as a set of two ODEs in space $z$.
  \item Apply normal mode expansion with $N_w$ flexural modes and $N_\phi$ torsional modes and recast the equations of motion as an algebraic matrix eigenvalue problem of dimension $N=N_w+N_\phi$ that depends on airspeed $U$.
\end{enumerate}

\begin{resultsbox}
\centering
\includegraphics[width=\linewidth]{figures/B.a.png}

\vspace{0.5em}
\small\textit{Part B(a): free-body diagram showing elastic axis $P$, mass axis $C$ ($x_{CP}$ aft), aerodynamic axis $Q$ ($x_{QP}$ aft), lift $L(U)$, pitching moment, internal shear $V$, and torque $T$ over $dz$.}
\end{resultsbox}

\begin{stepbox}
\textbf{Set conventions and kinematics}\\
Since $z$ points spanwise from the root and $w(z,t)$ is positive downward, a nose-up twist $\phi(z,t)$ is positive clockwise (as in Part A). The mass center $C$ is offset by $x_{CP}$ from the elastic axis $P$, so its vertical acceleration is $\ddot{w}-x_{CP}\ddot{\phi}$. The aerodynamic center $Q$ is offset by $x_{QP}$ from $P$.
\end{stepbox}

\begin{stepbox}
\textbf{Internal force/torque relations}\\
Since Euler--Bernoulli bending and Saint--Venant torsion apply,
\[
M=-EI\,w'',\qquad V=\frac{dM}{dz}=EI\,w''',\qquad T=GJ\,\phi',\qquad \frac{dT}{dz}=GJ\,\phi''.
\]
These relations let us replace $dV$ and $dT$ by derivatives of $w$ and $\phi$.
\end{stepbox}

\begin{stepbox}
\textbf{Aerodynamic linearization}\\
Since small perturbations are assumed, the sectional lift is linearized as
\[
L(U)=L_0(U)\,\phi,
\]
acting upward at $Q$; the associated aerodynamic moment about $P$ is $L_0(U)\,x_{QP}\,\phi$ (nose-up positive).
\end{stepbox}

\begin{stepbox}
\textbf{Part B(b): plunge equilibrium (force in $y$)}\\
Since internal shear varies as $V(z+dz)=V+dV$ and $V=EI\,w'''$, and the mass center acceleration is $\ddot{w}-x_{CP}\ddot{\phi}$, summing forces in $y$ over $dz$ gives
\[
V-(V+dV)+L(U)\,dz+m\,dz\bigl(\ddot{w}-x_{CP}\ddot{\phi}\bigr)=0.
\]
Since $dV/dz=EI\,w''''$, dividing by $dz$ yields the plunge PDE
\[
m\,\ddot{w}-m x_{CP}\,\ddot{\phi}+EI\,w''''+L_0(U)\,\phi=0.
\]
\end{stepbox}

\begin{stepbox}
\textbf{Part B(b): pitch equilibrium (moment about elastic axis $P$)}\\
Since torsion varies as $T(z+dz)=T+dT$ with $T=GJ\,\phi'$, and lift acts at $Q$ giving $L_0(U)x_{QP}\phi$, taking moments about $P$ over $dz$:
\[
T-(T+dT)-m x_{CP}\,dz\,\ddot{w}+I_p\,dz\,\ddot{\phi}-L_0(U)\,x_{QP}\,\phi\,dz=0.
\]
Using $dT/dz=GJ\,\phi''$ and dividing by $dz$ gives the pitch PDE
\[
-m x_{CP}\,\ddot{w}+I_p\,\ddot{\phi}-GJ\,\phi''-L_0(U)\,x_{QP}\,\phi=0.
\]
\end{stepbox}

\begin{resultsbox}
\textbf{Part B(b): coupled plunge--pitch PDEs}\\[0.6em]
\[
m\,\ddot{w}-m x_{CP}\,\ddot{\phi}+EI\,w''''+L_0(U)\,\phi = 0,
\qquad
-m x_{CP}\,\ddot{w}+I_p\,\ddot{\phi}-GJ\,\phi''-L_0(U)\,x_{QP}\,\phi = 0.
\]
\end{resultsbox}

\begin{stepbox}
\textbf{Part B(c): assume harmonic time factor $e^{st}$}\\
Since \(w(z,t)=\hat{w}(z)e^{st}\) and \(\phi(z,t)=\hat{\phi}(z)e^{st}\), we replace \( \dot{} \rightarrow s\) and obtain ODEs in $z$:
\[
m s^2 \hat{w}-m x_{CP} s^2 \hat{\phi}+EI\,\hat{w}''''+L_0(U)\,\hat{\phi} = 0,
\]
\[
-m x_{CP} s^2 \hat{w}+I_p s^2 \hat{\phi}-GJ\,\hat{\phi}''-L_0(U) x_{QP}\,\hat{\phi} = 0.
\]
\end{stepbox}

\begin{resultsbox}
\textbf{Part B(c): time-harmonic spanwise ODEs}\\[0.4em]
\[
m s^2 \hat{w}-m x_{CP} s^2 \hat{\phi}+EI\,\hat{w}''''+L_0(U)\,\hat{\phi} = 0,
\qquad
-m x_{CP} s^2 \hat{w}+I_p s^2 \hat{\phi}-GJ\,\hat{\phi}''-L_0(U) x_{QP}\,\hat{\phi} = 0.
\]
\end{resultsbox}

\begin{stepbox}
\textbf{Part B(d): normal-mode expansion and projections}\\
Since the uncoupled mode families are orthonormal, expand
\[
w(z,t)=\sum_{j=1}^{N_w}\eta_j^{w}(t)W_j(z),\qquad
\phi(z,t)=\sum_{j=1}^{N_\phi}\eta_j^{\phi}(t)\Phi_j(z).
\]
Project the $s$-domain ODEs onto $W_p$ and $\Phi_p$; using $\int_0^l W_p W_q\,m\,dz=\delta_{pq}$ and $\int_0^l \Phi_p\Phi_q\,dz=\delta_{pq}$ gives
\[
[m^{ww}]\,\ddot{\eta}^w + [m^{w\phi}]\,\ddot{\eta}^\phi + [k^{ww}_s]\,\eta^w + [k^{w\phi}_A(U)]\,\eta^\phi = 0,
\]
\[
[m^{\phi w}]\,\ddot{\eta}^w + [m^{\phi\phi}]\,\ddot{\eta}^\phi + [k^{\phi\phi}_s]\,\eta^\phi + [k^{\phi\phi}_A(U)]\,\eta^\phi = 0.
\]
Here
\[
[m^{ww}]_{p q}=m\int_0^l W_p W_q\,dz,\quad
[m^{w\phi}]_{p q}=-m x_{CP}\int_0^l W_p \Phi_q\,dz,
\]
\[
[m^{\phi\phi}]_{p q}=I_p\int_0^l \Phi_p \Phi_q\,dz,
\]
\[
[k^{ww}_s]_{p q}=EI\int_0^l W_p'''' W_q\,dz,\quad
[k^{\phi\phi}_s]_{p q}=-GJ\int_0^l \Phi_p'' \Phi_q\,dz,
\]
\[
[k^{w\phi}_A(U)]_{p q}= \int_0^l L_0(U)\,\Phi_q W_p\,dz,\quad
[k^{\phi\phi}_A(U)]_{p q}= -\int_0^l L_0(U)\,x_{QP}\,\Phi_q \Phi_p\,dz.
\]
\end{stepbox}

\begin{resultsbox}
\textbf{Part B(d): eigenvalue form}\\[0.4em]
Assuming \(e^{st}\) for modal coordinates yields the algebraic problem
\[
\Big[s^2
\begin{bmatrix}
[m^{ww}] & [m^{w\phi}]\\
[m^{\phi w}] & [m^{\phi\phi}]
\end{bmatrix}
+ \begin{bmatrix}
[k^{ww}_s] & 0\\
0 & [k^{\phi\phi}_s]
\end{bmatrix}
+ \begin{bmatrix}
[k^{w\phi}_A(U)]\\
[k^{\phi\phi}_A(U)]
\end{bmatrix}
\Big]
\begin{bmatrix}\eta^w\\ \eta^\phi\end{bmatrix}
= \mathbf{0}.
\]
\end{resultsbox}

\question{C. Ground vibration test (GVT) simulation for $N_w=4$, $N_\phi=3$}\label{quest:C}
\begin{tcolorbox}[colback=white,colframe=USC_Black90,title=Problem Assumption,sharp corners,boxrule=0.8pt]
  Assume $U=0$.
\end{tcolorbox}
\begin{enumerate}[label=(\alph*)]
  \item Display the number of modes $N_w$, $N_\phi$ of the uncoupled problems and calculate and display the total number of modes $N$ of the coupled problem.
  \item Calculate and display the coupled flexural and torsional frequencies and eigenvectors. Discuss your results.
  \item Calculate and plot the coupled modeshapes and discuss your results.
\end{enumerate}

\begin{resultsbox}
\centering
\includegraphics[width=\linewidth]{figures/C.ab.png}

\vspace{0.5em}
\small\textit{Part C(a)--(b): summary of uncoupled counts ($N_w=4$, $N_\phi=3$, $N=7$), coupled frequencies/eigenvalues, and key numerical outputs from the GVT simulation at $U=0$.}
\end{resultsbox}

\begin{resultsbox}
\textbf{Part C(c): coupled modeshapes ($N=7$)}\\[0.6em]
\begin{center}
\includegraphics[width=0.48\linewidth]{figures/C_mode1.png}\hfill
\includegraphics[width=0.48\linewidth]{figures/C_mode2.png}\\[0.4em]
\includegraphics[width=0.48\linewidth]{figures/C_mode3.png}\hfill
\includegraphics[width=0.48\linewidth]{figures/C_mode4.png}\\[0.4em]
\includegraphics[width=0.48\linewidth]{figures/C_mode5.png}\hfill
\includegraphics[width=0.48\linewidth]{figures/C_mode6.png}\\[0.4em]
\includegraphics[width=0.48\linewidth]{figures/C_mode7.png}
\end{center}
\small\textit{Coupled flexural/torsional shapes from the GVT case ($U=0$) for modes 1–7 in order of increasing frequency.}
\end{resultsbox}

\begin{resultsbox}
\textbf{Part C discussion}\\[0.5em]
Modes 1–2 are predominantly flexural; modes 3–4 show bending–torsion coupling; modes 5–7 are torsion-dominated. The slight shifts from the uncoupled frequencies reflect weak coupling at $U=0$, consistent with the $EI/GJ$ tuning in Parts A.1 and A.2. These mixed modes identify which shapes will interact first as airspeed grows toward flutter in Part D.
\end{resultsbox}

\question{D. Flutter eigenvalue analysis for $N_w=4$, $N_\phi=3$}\label{quest:D}
\begin{enumerate}[label=(\alph*)]
  \item Let airspeed $U=0,\dots,12~\mathrm{m/s}$ with 1001 steps. Plot frequencies and damping vs airspeed and find the flutter speed $U_F$ on the plots. Recall from HW01 the rigid airfoil flutter speed $U_F^{\text{rigid}}$ and discuss the results comparatively.
  \item Calculate frequencies, damping, and eigenvectors around flutter speed, i.e., at $U=(1-\varepsilon)U_F$, $U_F$, $(1+\varepsilon)U_F$, $\varepsilon=1\%$, and discuss your results.
  \item Plot flutter modeshapes at $U_F$ and discuss your results.
\end{enumerate}

\begin{resultsbox}
\centering
\includegraphics[width=\linewidth]{figures/D.a.png}

\vspace{0.4em}
\small\textit{Part D(a): frequency and damping trends vs airspeed $U$ (0--12 m/s) used to read flutter onset.}
\end{resultsbox}

\begin{resultsbox}
\centering
\includegraphics[width=\linewidth]{figures/D.b.png}

\vspace{0.4em}
\small\textit{Part D(b): zoomed view around $U_F$ highlighting the first coalescing root and sign change in damping.}
\end{resultsbox}

\begin{resultsbox}
\centering
\includegraphics[width=0.48\linewidth]{figures/D.c1.png}\hfill
\includegraphics[width=0.48\linewidth]{figures/D.c2.png}\\[0.4em]
\includegraphics[width=0.48\linewidth]{figures/D.c3.png}\hfill
\includegraphics[width=0.48\linewidth]{figures/D.c4.png}\\[0.4em]
\includegraphics[width=0.48\linewidth]{figures/D.c5.png}\hfill
\includegraphics[width=0.48\linewidth]{figures/D.c6.png}

\vspace{0.5em}
\small\textit{Part D(c): flutter modeshapes/eigenvectors near $U_F$ for the critical pair and higher modes.}
\end{resultsbox}

\begin{resultsbox}
\textbf{Part D discussion}\\[0.5em]
The frequency/damping map in D(a) shows a single flexural–torsional pair converging as $U$ increases; D(b) confirms the damping crosses zero at the same $U_F$, marking flutter onset. The mode snapshots in D(c) reveal that the critical pair mixes the first torsional with a higher flexural component—consistent with the weak but nonzero coupling seen in Part C. Because the damping slope near $U_F$ is steep, small aerodynamic or stiffness changes will shift $U_F$ noticeably, so the GVT correlation (Parts A–C) is essential for credible flutter prediction.
\end{resultsbox}

\question{E.1 Flexural vibration analysis (extra credit)}\label{quest:E1}
Derive the equations of flexural vibration for frequencies and modeshapes.

\begin{stepbox}
\textbf{E.1 Step 1: Euler--Bernoulli governing PDE}\\
Since slender wings obey Euler--Bernoulli theory, the transverse displacement $w(z,t)$ satisfies
\[
EI\,w''''(z,t)+m\,\ddot{w}(z,t)=0
\]
with $EI,m$ uniform along $0\le z\le l$.
\end{stepbox}

\begin{stepbox}
\textbf{E.1 Step 2: fixed--free boundary conditions}\\
Since the root is clamped: $w(0,t)=0,\ w'(0,t)=0$. Since the tip is free: $w''(l,t)=0,\ w'''(l,t)=0$.
\end{stepbox}

\begin{stepbox}
\textbf{E.1 Step 3: separation of variables}\\
Since $w(z,t)=\hat w(z)e^{st}$, substitution gives the spatial ODE
\[
\hat w''''-\beta^4\,\hat w=0,\qquad \beta^4=\frac{m s^2}{EI}.
\]
Applying the four BCs yields the characteristic relation
\[
\cosh(\beta l)\cos(\beta l)+1=0.
\]
\end{stepbox}

\begin{resultsbox}
\textbf{E.1 Roots of $\cosh\beta l\cos\beta l+1=0$}\\[0.4em]
\[
\beta_1 l=1.8751,\quad \beta_2 l=4.6941,\quad \beta_3 l=7.8548,\quad \beta_4 l=10.9955,\dots
\]
\small These agree with classical tabulations (e.g., Timoshenko \& Young, \textit{Vibration Problems in Engineering}, 5th ed.).
\end{resultsbox}

\begin{stepbox}
\textbf{E.1 Step 4: mode shapes and normalization}\\
Since tip compatibility enforces
\[
B_j=\frac{\sinh\beta_j-\sin\beta_j}{\cosh\beta_j+\cos\beta_j},
\]
the $j$th mode shape is
\[
W_j(z)=A_j\Big[(\cosh\beta_j z/l-\cos\beta_j z/l)-B_j(\sinh\beta_j z/l-\sin\beta_j z/l)\Big],
\]
with $A_j=1/\sqrt{l}$ to set unit modal mass: $\int_0^l m W_j W_k\,dz=\delta_{jk}$.
\end{stepbox}

\begin{resultsbox}
\textbf{E.1 Flexural natural frequencies}\\[0.4em]
\[
\omega_j=\beta_j^2\sqrt{\frac{EI}{m\,l^4}},\qquad f_j=\frac{\omega_j}{2\pi}.
\]
\end{resultsbox}

\begin{resultsbox}
\textbf{E.1 Python spot–check (unit beam, $l=1$, $EI=m=1$)}\\[-0.2em]
\begin{verbatim}
flexural betas: [1.87510407 4.69409113 7.85475744 10.99554073]
\end{verbatim}
\[
f_1=\frac{\beta_1^2}{2\pi}=0.560,\qquad f_2=2.645,\qquad f_3=6.864.
\]
\small Computed with mpmath findroot on \(\cosh\beta\cos\beta+1=0\) (script available on request).
\end{resultsbox}

\begin{resultsbox}
\centering
\includegraphics[width=\linewidth]{figures/E1_modes.png}

\vspace{0.4em}
\small\textit{E.1 visualization: first three cantilever flexural modes (normalized, offset for clarity).}
\end{resultsbox}

\clearpage

\question{E.2 Torsional vibration analysis (extra credit)}\label{quest:E2}
Derive the equations of torsional vibration for frequencies and modeshapes.

\begin{stepbox}
\textbf{E.2 Step 1: Saint--Venant torsion PDE}\\
Since uniform shafts twist without warping, the twist $\phi(z,t)$ obeys
\[
GJ\,\phi''(z,t)=I_0\,\ddot{\phi}(z,t),
\]
where $I_0$ is the polar mass moment per unit span.
\end{stepbox}

\begin{stepbox}
\textbf{E.2 Step 2: fixed--free boundary conditions}\\
Since root is clamped: $\phi(0,t)=0$. Since tip is free (zero torque): $\phi'(l,t)=0$.
\end{stepbox}

\begin{stepbox}
\textbf{E.2 Step 3: separation and eigenvalue condition}\\
Since $\phi(z,t)=\hat\phi(z)e^{st}$, spatial equation $\hat\phi''+\gamma^2\hat\phi=0$ with $\gamma^2=I_0 s^2/(GJ)$.
Applying BCs gives $\cos(\gamma l)=0 \Rightarrow \gamma_j l=(2j-1)\pi/2$.
\end{stepbox}

\begin{resultsbox}
\textbf{E.2 Torsional modes and frequencies}\\[0.4em]
\[
\Phi_j(z)=B_j\sin(\gamma_j z),\qquad B_j=\sqrt{\frac{2}{l}},\qquad \gamma_j=\frac{(2j-1)\pi}{2l},
\]
\[
\omega_j=\gamma_j\sqrt{\frac{GJ}{I_0}},\qquad f_j=\frac{\omega_j}{2\pi}.
\]
\small Fixed–free torsional spectrum follows directly from $\cos\gamma l=0$ (see Blevins, \textit{Formulas for Natural Frequency and Mode Shape}).
\end{resultsbox}

\begin{resultsbox}
\textbf{E.2 Python spot–check (unit rod, $l=1$, $GJ=I_0=1$)}\\[-0.2em]
\begin{verbatim}
torsion gammas*l: [1.57079633 4.71238898 7.85398163]
\end{verbatim}
\[
f_1=\frac{\gamma_1}{2\pi}=0.250,\qquad f_2=0.750,\qquad f_3=1.250.
\]
\end{resultsbox}

\begin{resultsbox}
\centering
\includegraphics[width=\linewidth]{figures/E2_modes.png}

\vspace{0.4em}
\small\textit{E.2 visualization: first three cantilever torsional modes (normalized).}
\end{resultsbox}

\clearpage
\appendix
\section*{Appendix A: MATLAB Source Listings}
\label{appendix:matlab}
\addcontentsline{toc}{section}{Appendix A: MATLAB Source Listings}

\lstinputlisting{Matlab/HW03_tors_flex_flutter_EXAMPLE.m}
\lstinputlisting{Matlab/sort_norm_eig.m}

\end{document}

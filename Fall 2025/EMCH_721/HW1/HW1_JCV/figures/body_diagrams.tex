% This file contains only the TikZ picture code.
% It can be included in any main LaTeX document.

\begin{tikzpicture}

% Third diagram (shifted right by another 9 units to avoid overlap)
    \begin{scope}[shift={(0,6)}]
        \tikzset{
            offset_line_vert/.style={thin, dashed, gray!70}, % Vertical dashed lines
            offset_arrow/.style={<->, thick, black},       % Double-headed arrows for offsets
            label_text/.style={font=\small, align=left},
        }
        % --- Define Coordinates for reference points ---
        \coordinate (LE) at (0.2,0);
        \coordinate (TE) at (7,0);
        % Points C and P along the chord line
        \coordinate (C_ref) at ($(LE)!0.45!(TE)$);
        \coordinate (P_ref) at ($(LE)!0.65!(TE)$);
        \coordinate (Q_ref) at ($(LE)!0.25!(TE)$);

        % --- Horizontal reference lines ---
        \draw[black, thin] (-0.5, 1.3) -- (7.5, 1.3); % Top line

        % --- Offset Labels ---
        \node[label_text, anchor=west] at (P_ref |- 0, 2.1) {aerodynamic offset};
        \node[label_text, anchor=west] at (P_ref |- 0, 1.8) {static offset};

        % --- x_QP Measurement ---
        % Vertical dashed lines from Q_ref and P_ref to the top measurement line
        \draw[offset_line_vert] (Q_ref |- 0, -4.5) -- (Q_ref |- 0, 2.5);
        \draw[offset_line_vert] (P_ref |- 0, -4.2) -- (P_ref |- 0, 2.5);
        % Double-headed arrow for x_QP
        \draw[offset_arrow] (Q_ref |- 0, 2.1) -- node[above, font=\small] {$x_{QP}$} (P_ref |- 0, 2.1);

        % --- x_CP Measurement ---
        % Vertical dashed lines from C_ref to the lower measurement line
        \draw[offset_line_vert] (C_ref |- 0, -4.2) -- (C_ref |- 0, 2.1);
        % Double-headed arrow for x_CP
        \draw[offset_arrow] (C_ref |- 0, 1.8) -- node[below, font=\small] {$x_{CP}$} (P_ref |- 0, 1.8);
    \end{scope}
    % First diagram (left position)
    \begin{scope}[shift={(0,2)}]
        \tikzset{
            foil/.style={thick, fill=blue!5!white, draw=black},
            accel_arrow/.style={-Latex, very thick, blue!80!black},
            label_text/.style={font=\sffamily\small, align=left},
            point/.style={circle, fill=black, inner sep=1.5pt},
        }
        % --- Define Coordinates ---
        \coordinate (LE) at (0.2,0);
        \coordinate (TE) at (7,0);
        \coordinate (TopMid) at (0.8, 0.6);
        \coordinate (BottomMid) at (0.8, -0.6);

        % Points Q, C, and P along the chord line
        \coordinate (Q) at ($(LE)!0.25!(TE)$);
        \coordinate (C) at ($(LE)!0.45!(TE)$);
        \coordinate (P) at ($(LE)!0.65!(TE)$);

        % --- Draw the Airfoil and Points ---
        \draw[foil] (TE) -- (BottomMid) arc (270:90:0.6) -- cycle;

        % Draw points Q, C, and P
        \node[point, label=right:$Q$] at (Q) {};
        \node[point, label=right:$C$] at (C) {};
        \node[point, label=right:$P$] at (P) {};

        % --- Draw Accelerations ---
        \draw[accel_arrow] (C) -- ($(C)+(0,-1.2)$) node[right=2pt] {$\ddot{u}_y = \ddot{h} - x_{CP} \ddot{\theta}$};

        % Angular acceleration (theta_dot_dot) around P, now dashed and reversed
        % Draws a shorter arc
        \draw[accel_arrow, dashed] ($(P)+({-0.7*cos(45)},{0.7*sin(45)})$) arc (150:20:0.7) node[midway, above] {$\ddot{\theta}$};

    \end{scope}

    % Second diagram (shifted right by 9 units to avoid overlap)
    \begin{scope}[shift={(0,5)}]
        \tikzset{
            foil/.style={thick, fill=blue!5!white, draw=black},
            force_arrow/.style={-Latex, very thick, black},
            accel_arrow/.style={-Latex, very thick, black, dashed},
            label_text/.style={font=\sffamily\small, align=left},
            point/.style={circle, fill=black, inner sep=1.5pt},
        }
        % --- Define Coordinates ---
        \coordinate (LE) at (0.2,0);
        \coordinate (TE) at (7,0);
        \coordinate (TopMid) at (0.8, 0.6);
        \coordinate (BottomMid) at (0.8, -0.6);

        % Points Q, C, and P along the chord line
        \coordinate (Q) at ($(LE)!0.25!(TE)$);
        \coordinate (C) at ($(LE)!0.45!(TE)$);
        \coordinate (P) at ($(LE)!0.65!(TE)$);

        % --- Draw the Airfoil and Points ---
        \draw[foil] (TE) -- (BottomMid) arc (270:90:0.6) -- cycle;

        % Draw points Q, C, and P
        \node[point, label=right:$Q$] at (Q) {};
        \node[point, label=right:$C$] at (C) {};
        \node[point, label=right:$P$] at (P) {};

        % --- Draw Forces and Accelerations ---

        % Lift Force (L) at Q
        \draw[force_arrow] (Q) -- ($(Q)+(0,1.5)$) node[right] {$L$};

        % Vertical Force (K_h*h) at P, with label at the bottom
        \draw[force_arrow] ($(P)+(0,-1.5)$) node[left] {$K_h h$} -- ($(P)-(0,0.1)$);

        % Acceleration (h_dot_dot) at C, pointing DOWN
        \draw[accel_arrow] ($(C)+(-0.3,0)$)  -- ($(C)+(-0.3,-1.2)$) node[left] {$\ddot{h}$};
        
        \draw[force_arrow] ($(P)+({0.65},{0.5*sin(45)})$) arc (20:150:0.7) node[midway, above] {$K_\theta \theta$};
    \end{scope}

\end{tikzpicture}
